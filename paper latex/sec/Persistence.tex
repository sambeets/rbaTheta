Frequency of an event can be defined as the number of times it occurs. Frequency of occurrences for $\Delta w_s$, $\Delta t$, $ \text{mean} \Delta w_s$, $\alpha \Delta w_s$ with corresponding ranges are evaluated. The ranges or bins are assumed to be linearly spaced between 1 and 100. The algorithmic representation for the RBA$_\theta$ frequency function is presented in algorithm \ref{algo2}.
The table \ref{tbl:freq1}  presents the start and end points for the event parameters for frequency evaluation. Input time series can be of different forms, for instance a triangular or step. For the former the persistence would be zero because there is only one peak for a time period. In case of the later there would be multiple points with or without slight variations with respect to a time period. 
Table \ref{tbl:freq2} presents the parameters and corresponding the frequency of occurrences of events.

\begin{algorithm}[!htbp] 
\SetKwInOut{Input}{Input}
\SetKwInOut{Output}{Output}
\SetAlgoLined
\Input{$\Delta w_s$, $\Delta t_s$, $(\theta (\Delta w_s))$, $\text{mean} (\Delta w_s)$}
\KwResult{$\Delta w_s^f$,
  $\Delta t_s^f$, $(\theta (\Delta w_s))^f$, $\text{mean} (\Delta w_s^f)$}
\For{bin$_1$=-1 $\ldots 1$}{
$w_{s,bin_1} $ 
}
\For {bin$_2$ = 1 $\ldots$ max $(\Delta t_s)$}{
$\Delta \; t_{s,bin_2}$
}
\For{bin$_3 = -90 \ldots 90$}{
$\theta ( \Delta w_{s, bin_3} )$ 
 }
%\For{$bin_3$ = (min ( mean} $(\Delta w_s))  \ldots $ max(mean $( \Delta w_s))$ }{
%$mean (\Delta w_{s, bin_3} $ 
% }
%%%%%%%%%%%%  bin3 and bin4 ? 
% frequency . what is a parameter?
\underline{function} frequency \\
\Input{param, bin} 
\For{ $i \leftarrow$ \textbf{to} \textit{length} (bin) }{
\For{ $k \leftarrow$ \textbf{to} \textit{length} (param)}{
\If{ param[k] $\leq$  bin[k+1]}{
$f[i] += 1$}
}}
\For{ $i \leftarrow$ \textbf{to} \textit{length} (bins)-1}{
\For{ $k \leftarrow$ \textbf{to} \textit{length} (param)}{
frequency[i] = f[i]
}}
$ \Delta w_{s}^f $ = frequency ($\Delta w_s , \Delta w_{s,bin_1} $) \\
$ \Delta t_{s}^f $ = frequency ($\Delta t_s , \Delta t_{s,bin_2}$) \\
$ \theta (\Delta w_{s}^f) $ = frequency ($ \theta (\Delta w_s , \Delta w_{s,bin_3})$) \\
$ mean (\Delta w_{s}^f) $ = frequency ($ mean (\Delta w_s , \Delta w_{s,bin_4})$) \\
 \caption{$RBA_\theta$ frequency function}
 \label{algo2}
\end{algorithm}


\begin{table}[!htbp]
\centering
\begin{tabular}{ccc} \hline
Event parameters 	& Starting point & Ending point \\ \hline
$\Delta w_s$ 			& -1 					& 01 					\\
$\Delta t$ 				& 01 					& max$(\Delta t)$ \\
mean$(\Delta w_s)$ & min(mean$(\Delta w_s)$) & max(mean$(\Delta w_s)$) \\
$\alpha (\Delta w_s)$ & -90$^o$ & 90$^o$ \\ \hline
\end{tabular}
\caption{Frequency measures of events with start and end points}
\label{tbl:freq1}
\end{table}



\begin{table}[!htbp]
\centering
\begin{tabular}{ccccccccc} \hline
Event & $\Delta w_s$ &  $\Delta w_s^f$  & $\Delta t$ & $\Delta t^f$ & $\theta \Delta w_s$ & $\theta \Delta w_s^f$ & mean($\Delta w_s)$ & mean($\Delta w_s^f$) \\ \hline
01	&	0.218	&	14	&	1	&	1	&	65.340	&	1	&	0.689	&	5	\\
02	&	-0.355	&	10	&	109	&	9	&	-1.868	&	167	&	0.514	&	5	\\
03	&	0.201	&	20	&	42	&	13	&	2.743	&	37	&	0.583	&	3	\\
04	&	-0.558	&	7	&	169	&	5	&	-1.891	&	167	&	0.405	&	9	\\
05	&	-0.415	&	4	&	71	&	16	&	-3.347	&	23	&	0.333	&	9	\\
06	&	0.241	&	18	&	67	&	17	&	2.063	&	159	&	0.246	&	4	\\
07	&	0.468	&	4	&	118	&	8	&	2.270	&	159	&	0.449	&	4	\\
08	&	-0.168	&	26	&	57	&	15	&	-1.685	&	167	&	0.599	&	2	\\
09	&	-0.231	&	16	&	57	&	15	&	-2.320	&	167	&	0.407	&	9	\\
10	&	0.422	&	9	&	69	&	12	&	3.503	&	37	&	0.502	&	3	\\
\hline
\end{tabular}
\caption{RBA$_\theta$ frequency of events}
\label{tbl:freq2}
\end{table}


The persistence can be defined as the act of continuing to exist past the usual time. Therefore the persistence event is a series of points that stays within the defined threshold value for longer than a predefined time limit. The algorithmic representation of persistence is presented in algorithm \ref{algo3}. The time period is chosen to be two days. In this period all the events extracted are sorted and the first event that has persisted for highest time is saved as a persistent event. The table \ref{tbl:p-rba} presents the persistence values for ten events. The persistence values for $w_{s,t}$ and $w_{s, t+ \Delta t}$ are presented through time $t$, time difference $t+ \Delta t$, mean and time difference $\Delta t$.

\begin{algorithm}[!htbp] 
\SetKwInOut{Input}{Input}
\SetKwInOut{Output}{Output}
\SetAlgoLined
\Input{ $mean, \Delta t$}
\KwResult{ $\overline{mean}, \overline{\Delta t}$}
periods=$t+1$ \\
bins= linspace(0,$t$,periods)  \\
t = length(filtered) \\
\For{$i$ in range(length(bins-1))}{
\For{$k$ in range(length($mean, \Delta t$))}{
\If{bins$_i <$ $mean_k , \Delta t_k$$<$ bins$_{i+1}$}{
sort ($\overline{mean}, \overline{\Delta t}$)
}}}
\caption{$RBA_\theta$ persistence function}
 \label{algo3}
\end{algorithm}

\begin{table}[!htbp]
\centering
\begin{tabular}{ccccccc} \hline
Event & $w_{s,t}$ &  $w_{s, t+ \Delta t}$  & $t$ & $t + \Delta t$ & mean & $\Delta t$ \\ \hline
01	&	0.453	&	0.302	&	100	&	477	&	0.378	&	377	\\
02	&	0.439	&	0.591	&	102	&	832	&	0.515	&	730	\\
03	&	0.265	&	0.416	&	1056	&	1389	&	0.340	&	333	\\
04	&	0.583	&	0.411	&	1624	&	1999	&	0.497	&	375	\\
05	&	0.452	&	0.302	&	100	&	477	&	0.377	&	377	\\
06	&	0.438	&	0.590	&	102	&	831	&	0.514	&	729	\\
07	&	0.276	&	0.428	&	1054	&	1392	&	0.352	&	338	\\
08	&	0.571	&	0.408	&	1622	&	1999	&	0.489	&	377	\\
09	&	0.482	&	0.331	&	94	&	466	&	0.406	&	372	\\
10	&	0.471	&	0.621	&	96	&	889	&	0.546	&	793	\\
\hline
\end{tabular}
\caption{RBA$_\theta$ persistence of events}
\label{tbl:p-rba}
\end{table}




