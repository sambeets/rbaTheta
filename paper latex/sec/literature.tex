\subsection{Previous work} \label{previous work}

There are many studies in the literature on characteristics of wind power, correlations of wind and wind power, forecasting wind and wind power output, scenario generation. In \cite{bianco2016wind} the authors proposed a model to forecast ramp events as well. They modelled observed wind speeds into forecast models and converted this into power forecasts with the help of the power curve of the wind turbines. It suggests that the same method could be implemented for solar power plants. Another noteworthy one is a model-free forecast generation implemented with Generative Adversarial Networks (GAN). GANs have two de-convolutional, one that starts by generating random data and the out discriminates whether its input is coming from the generator or historical data. These two neural networks play a Nashville game while giving feedback to each other, both getting better over time until the generator generates data that is almost like a forecast so that the discriminator can not discriminate any more. \cite{cui2016optimized} proposed a probabilistic forecasting method, utilizing a Neural Network(NN) to generate possible future scenarios, employing an objective function based on cumulative distribution functions and auto-correlation functions to train the NN, primarily teaching it their distribution. Again another \cite{karatepe2013wind} proposed a model to synthesized wind speed scenarios based on statistical parameters of wind and Markov chains. In contrast, \cite{kaut2014copula} proposes a new heuristic to generate scenarios that use copulas instead of common correlation functions. \cite{ArticleNo1,ArticleNo2} introduced the terminology for identification of ramp events, ramping behaviour analysis(RBA), which comprises the perspective used in this study. They also filtered and extracted events, and clustered them into groups. More studies have been performed on identifying ramp events in \cite{bossavy2013forecasting, bossavy2013novel, cui2016optimized}. 

\subsection{Contribution of this paper}
This paper investigates the variations in the wind power generation and contributes to the variation quantification and prediction methodology. The proposed model can be used for operational planning of wind farms and injection of wind power into the system.
RBA$_\theta$ is a novel method considering spatio-temporal information of power variations. The proposed model has the following advantages over the existing models: 
\begin{description}
\item[Cleaning the data] Prediction of discrete time-series data may contain some irregularities in form of noise. Therefore, pre-processing the data and removing the noise content is essential for accurate prediction. 
\item[Spatial information] The location of the turbines has a significant impact of the production. Therefore, RBA$_\theta$ takes into account the spatio-temporal information: location of turbines and power production through a markovian transition matrix. This matrix has transition probabilities with likelihood from one state to another state.
\item[Prediction of events] The operational planning for a wind farm plans for responses to significant events such as high power generation and very low power generation. As a consequence, generating the scenarios for the events that might occur based on the historical pattern facilitates the planning.
\end{description}
 