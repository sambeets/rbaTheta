\section{Introduction} \label{sec:intro}

Wind energy is stochastic in nature due the flow of wind is a product of multiple natural phenomenon. 
When there is a small penetration of wind into the power systems, the uncertain behaviour of the wind power generation is treated as just another uncertainty on the demand side. Moreover, the conventional power stations covering this variability, require additional energy and reduce the environmental benefits. Forecasting is one of the many possible solutions to this problem, as well as an interconnected grid, energy storage technologies, demand-side management such as electric vehicles. Forecasting aims to model the uncertainties inherited by the grid through wind power production and thus is a necessary and cost-effective element for the optimal integration of wind power into energy systems. However, forecasting is never accurate and literature suggests providing bounds for the forecasts or confidence intervals.
Wind power is intermittent by nature due to weather. A wind power ramp event can be termed as a sudden change in the output power over a set threshold. Mathematically, the absolute difference between power produced $P_t$ in time $t$ and $(t + \Delta \; t)$ that is above the set threshold $\overline{P}$ is a ramp event as in \eqref{eqn:rba}. However the threshold value is subjective. 

\begin{equation} \label{eqn:rba}
    | P_{(t + \Delta \; t)} - P_t | > \overline{P}
\end{equation}

\par
The system operator (SO) has to keep the system balanced i.e, the generation must meet the demand at each point in time. Wind ramp events can be positive or negative based on the generation swings. If positive, then the wind turbine has to shut down to avoid accidents or damage to the system whereas if the swing is negative the SO has to find a replacement to mitigate the demand. From economical point of view, both the energy not used and energy from an alternative resource are crucial. 
Each wind farm forecasts wind speed and power production from historical data over time with an objective to determine potential investment and operations. In long-term forecasting the events become insignificant due to time-stretch while in short-term forecasting of events are more accurate. Furthermore, the time interval $\Delta \; t$ is typically 10 minutes for ramp events. The $\overline{P}$ is either set to an absolute value for a wind park or a certain percentage of the generation depending on the installed capacity. The problem with this practice is that the peak generation capacity varies through seasons and turbine maintenance or new installations. Although the threshold is subjective to peculiarities of a wind park, the methods to classify ramp events is generic. This work draws its focus on the procedure to detect ramp events. In addition demonstrates the application on real data from wind park. 